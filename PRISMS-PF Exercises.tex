\documentclass[]{article}
\usepackage{lmodern}
\usepackage{amssymb,amsmath}
\usepackage{ifxetex,ifluatex}
\usepackage{fixltx2e} % provides \textsubscript
\ifnum 0\ifxetex 1\fi\ifluatex 1\fi=0 % if pdftex
  \usepackage[T1]{fontenc}
  \usepackage[utf8]{inputenc}
\else % if luatex or xelatex
  \ifxetex
    \usepackage{mathspec}
  \else
    \usepackage{fontspec}
  \fi
  \defaultfontfeatures{Ligatures=TeX,Scale=MatchLowercase}
\fi
% use upquote if available, for straight quotes in verbatim environments
\IfFileExists{upquote.sty}{\usepackage{upquote}}{}
% use microtype if available
\IfFileExists{microtype.sty}{%
\usepackage[]{microtype}
\UseMicrotypeSet[protrusion]{basicmath} % disable protrusion for tt fonts
}{}
\PassOptionsToPackage{hyphens}{url} % url is loaded by hyperref
\usepackage[unicode=true]{hyperref}
\hypersetup{
            pdfborder={0 0 0},
            breaklinks=true}
\urlstyle{same}  % don't use monospace font for urls
\usepackage{graphicx,grffile}
\makeatletter
\def\maxwidth{\ifdim\Gin@nat@width>\linewidth\linewidth\else\Gin@nat@width\fi}
\def\maxheight{\ifdim\Gin@nat@height>\textheight\textheight\else\Gin@nat@height\fi}
\makeatother
% Scale images if necessary, so that they will not overflow the page
% margins by default, and it is still possible to overwrite the defaults
% using explicit options in \includegraphics[width, height, ...]{}
\setkeys{Gin}{width=\maxwidth,height=\maxheight,keepaspectratio}
\IfFileExists{parskip.sty}{%
\usepackage{parskip}
}{% else
\setlength{\parindent}{0pt}
\setlength{\parskip}{6pt plus 2pt minus 1pt}
}
\setlength{\emergencystretch}{3em}  % prevent overfull lines
\providecommand{\tightlist}{%
  \setlength{\itemsep}{0pt}\setlength{\parskip}{0pt}}
\setcounter{secnumdepth}{0}
% Redefines (sub)paragraphs to behave more like sections
\ifx\paragraph\undefined\else
\let\oldparagraph\paragraph
\renewcommand{\paragraph}[1]{\oldparagraph{#1}\mbox{}}
\fi
\ifx\subparagraph\undefined\else
\let\oldsubparagraph\subparagraph
\renewcommand{\subparagraph}[1]{\oldsubparagraph{#1}\mbox{}}
\fi

% set default figure placement to htbp
\makeatletter
\def\fps@figure{htbp}
\makeatother


\date{}

\begin{document}

PRISMS-PF Training Exercises:

Here is a set of exercises to familiarize yourself with PRISMS-PF. Most
users find that these problems take several hours to complete in a
training environment where questions can be answered in real time. The
problems are approximately in ascending order of difficulty. We
recommend copying and renaming the example application directories
before making modifications so that you still have the original versions
to refer to. Delete the file ``CMakeCache.txt'' in the newly created
directory.

\begin{enumerate}
\def\labelenumi{\arabic{enumi}.}
\item
  \textbf{Boundary Conditions I: }

  \emph{Changing boundary conditions for the Allen-Cahn example problem}

  \begin{enumerate}
  \def\labelenumii{\alph{enumii}.}
  \item
    Change the BCs in the Allen-Cahn application to zero flux (the
    natural BC) on the top boundary, eta = 0 on the bottom boundary (a
    Dirichlet BC), and periodic on the two side boundaries (see diagram
    below)

    \includegraphics[width=3.28237in,height=2.49537in]{media/image1.png}
  \end{enumerate}
\item
  \textbf{Adaptivity: }
\end{enumerate}

\begin{quote}
\emph{Add adaptivity to the Allen-Cahn example problem}
\end{quote}

\begin{enumerate}
\def\labelenumi{\alph{enumi}.}
\item
  Copy the adaptivity section of parameters.in from the
  cahnHilliardWithAdaptivity application into the Allen-Cahn
  application, changing the variable that determines the mesh adaptivity
  from ``c'' to ``n''. See how much you can decrease the run time by
  using adaptivity, adjusting the parameters so that the solution
  doesn't change significantly. (Note that the initial refine factor
  should be between the max and min levels of refinement.)
\item
  Repeat with the minimum mesh size decreased by a factor of 2 in each
  direction. For stability purposes, the time step will need to be
  decreased by a factor of 4.
\end{enumerate}

\begin{enumerate}
\def\labelenumi{\arabic{enumi}.}
\item
  \textbf{Postprocessing:}
\end{enumerate}

\begin{quote}
\emph{Use the PRISMS-PF postprocessor to verify that the integral of a
field is conserved. (Note that this is a bit of a degenerate
postprocessed field, typically the postprocessed fields would be
different than the ``primary'' fields listed in equations.h. We're
adding one of the primary fields as a postprocessed field because only
postprocessed fields can be integrated.)}
\end{quote}

\begin{enumerate}
\def\labelenumi{\alph{enumi}.}
\item
  Add a second postprocessing variable in the cahnHilliard application
  for the concentration
\item
  Run a simulation and verify that the integral of the concentration is
  constant by examining the output in the file ``integratedFields.txt''.
\end{enumerate}

\begin{enumerate}
\def\labelenumi{\arabic{enumi}.}
\item
  \textbf{Initial Conditions and Computational Domain: }

  \emph{Adding a third particle to the coupled Allen-Cahn/Cahn-Hilliard
  application}
\end{enumerate}

\begin{enumerate}
\def\labelenumi{\alph{enumi}.}
\item
  Increase the size of the computational domain by 50\% in each
  direction, using the same mesh size as in the original calculation
  (i.e. you will need to increase the total number of elements from 192
  in each direction to 288).
\item
  Add a third particle to the initial condition in a location of your
  choice that doesn't overlap with either of the other two particles.
  (Note: if the particles do overlap, the simulation will likely crash.
  However, you should be able to open the initial output in VisIt.)
\end{enumerate}

\begin{enumerate}
\def\labelenumi{\arabic{enumi}.}
\item
  \textbf{Model constants:}
\end{enumerate}

\emph{Add more model constants to an application}

\begin{enumerate}
\def\labelenumi{\alph{enumi}.}
\item
  In the allenCahn application, the derivative of the free energy is
  hardcoded in equations.h through the variable ``fnV''. The
  corresponding free energy expression is:
\end{enumerate}

\[f_{\text{chem}} = \eta^{4} - 2\eta^{3} + \eta^{2}\]

\begin{enumerate}
\def\labelenumi{\alph{enumi}.}
\item
  Make the necessary modifications to parameters.in, equations.h, and
  customPDE.h to turn this into a generic 4\textsuperscript{th} order
  polynomial with coefficients set in parameters.in as more ``Model
  constants''.
\item
  Also make the necessary changes in postprocess.h so that the free
  energy expression is correct for outputted total free energy
\item
  Note how the solution changes when the homogenous free energy is
  changed to:
\end{enumerate}

\[f_{\text{chem}} = 4\eta^{4} - 8\eta^{3} + 4\eta^{2}\]

by changing the constants in parameters.in

\begin{enumerate}
\def\labelenumi{\arabic{enumi}.}
\item
  \textbf{Boundary Conditions II: }

  \emph{Strained precipitate evolution}

  \begin{enumerate}
  \def\labelenumii{\alph{enumii}.}
  \item
    Run the un-modified precipitateEvolution example and move the output
    files into a new directory named ``no\_strain''.
  \item
    Change the boundary conditions for the precipitate evolution
    application so that the x=40 boundary is displaced by +1 units, and
    that the boundary condition on the x component of the displacement
    along the y=0 and y=40 boundaries are zero derivative (see diagram
    below). Move the output files to a new directory named ``tension''
    and compare the results to the unstrained results.
  \item
    Repeat, but with the x=40 boundary displaced by -0.2 units, moving
    the output files to a new directory named ``compression''.
  \item
    Change the boundary condition on the x=40 to a Dirichlet BC that
    increases from 0 to +1 over the course of the simulation (Hints: use
    a NON\_UNIFORM\_DIRICHLET boundary condition, and the current time
    can be accessed in non-uniform Dirichlet BC function via the
    variable ``time'')

    \includegraphics[width=3.64376in,height=2.50000in]{media/image2.png}
  \end{enumerate}
\item
  \textbf{Governing Equations I: }

  \emph{Add a barrier term to the Precipitate Evolution application}

  \begin{enumerate}
  \def\labelenumii{\alph{enumii}.}
  \item
    Move two precipitates with different order parameters closer
    together and observe what happens when they touch (space them far
    apart enough that it takes a few thousand time steps for this to
    happen).
  \item
    Add a penalty term for overlapping structural order parameters. The
    barrier in the free energy is:
  \end{enumerate}
\end{enumerate}

\[f_{\text{barrier}} = n_{1}^{2}n_{2}^{2} + n_{1}^{2}n_{3}^{2} + n_{2}^{2}n_{3}^{2} + n_{1}^{2}n_{2}^{2}n_{3}^{2}\]

\begin{quote}
In each Allen-Cahn equation, this adds a term to the ``rniV'' residual
equal to
\(\Delta t\ M\ \frac{\partial f_{\text{barrier}}}{\partial n_{i}}\),
where ``i'' is the index of the structural order parameter.
\end{quote}

\begin{enumerate}
\def\labelenumi{\alph{enumi}.}
\item
  Add the barrier term to the calculation of the total free energy, note
  how much it shifts the energy.
\end{enumerate}

\begin{enumerate}
\def\labelenumi{\arabic{enumi}.}
\item
  \textbf{Governing Equations II: }

  \emph{Create a new application that simulates the growth of particle
  with a kinetically determined faceted shape. Similar models have been
  used for simulating selective area epitaxy and electrodeposition.}

  \begin{enumerate}
  \def\labelenumii{\alph{enumii}.}
  \item
    Create a new application by copying the
    ``cahnHilliardWithAdaptivity'' application and renaming it
    ``kineticWulffShape''. Delete the ``CMakeCache.txt'' file.
  \item
    Change the initial condition to a circle with radius of 12 units in
    the center of the domain. Run a (short) simulation to check that the
    initial condition is correct.
  \item
    Add a source term to the ``rcV'' residual:
    0.05*McV*userInputs.dtValue* std::max(c*(1.0-c),constV(0)). This
    source term will add mass at the interface, causing the initial
    particle to grow. Run a simulation to make sure that the particle is
    growing.
  \item
    Multiply the source term by an anisotropy coefficient, \(\gamma\),
    where:
  \end{enumerate}
\end{enumerate}

\[\gamma\  = \ 1 + 0.8*\left\lbrack 4*\left( \left( \frac{\frac{\partial c}{\partial x}}{\left| \nabla c \right| + 1.0 \times 10^{- 10}} \right)^{4} + \left( \frac{\frac{\partial c}{\partial y}}{\left| \nabla c \right| + 1.0 \times 10^{- 10}} \right)^{4} \right) - 3 \right\rbrack\]

Run the simulation to see how the orientation-dependent source term
causes the morphology to change.

\begin{enumerate}
\def\labelenumi{\arabic{enumi}.}
\item
  \textbf{Computational Domain II} (Warning: Requires some C++
  knowledge)\textbf{:}
\end{enumerate}

\begin{quote}
\emph{Override the core PRISMS-PF function that creates the mesh to
create a circular domain}
\end{quote}

\begin{enumerate}
\def\labelenumi{\alph{enumi}.}
\item
  Add the following member function to customPDE for the allenCahn
  application (add it to the bottom of customPDE.h and add the function
  declaration to the ``Methods specific to this subclass'' section of
  the customPDE class declaration):
\end{enumerate}

\begin{quote}
template \textless{}int dim, int degree\textgreater{}

void
customPDE\textless{}dim,degree\textgreater{}::makeTriangulation(parallel::distributed::Triangulation\textless{}dim\textgreater{}
\& tria) const\{

GridGenerator::hyper\_ball (tria,
Point\textless{}dim\textgreater{}(userInputs.domain\_size{[}0{]},userInputs.domain\_size{[}1{]}/2.0),
userInputs.domain\_size{[}1{]}/2.0);

// Attach a spherical manifold to the semicircular part of the domain so
that it gets refined with rounded edges

static const SphericalManifold\textless{}dim\textgreater{}
boundary(Point\textless{}dim\textgreater{}(userInputs.domain\_size{[}0{]},userInputs.domain\_size{[}1{]}/2.0));

tria.set\_manifold(0,boundary);

\}
\end{quote}

This will override the ``makeTriangulation'' function in the
MatrixFreePDE class that can only create rectangular meshes. Instead,
this function uses the ``hyper\_ball'' mesh generator from deal.II that
creates circular or spherical domains. Other mesh shapes that deal.II
can generate are given here:

https://www.dealii.org/8.5.0/doxygen/deal.II/namespaceGridGenerator.html.

\begin{enumerate}
\def\labelenumi{\alph{enumi}.}
\item
  Reduce the refine factor in parameters.in to 6 (the coarsest mesh for
  a circular domain is still partially refined, reducing the amount we
  have to refine it further)
\end{enumerate}

\end{document}
